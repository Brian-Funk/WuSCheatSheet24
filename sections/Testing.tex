


\mysection[col9]{\centering Testing}


\DEF{Entscheidungsregel}{
Die Hypothese $H_0$ wird verworfen, wenn $T(\omega) \in K$,
Die Hypothese $\mathrm{H}_0$ wird nicht verworfen bzw. angenommen, wenn $T(\omega) \notin K$}


\DEF{Signifikanzniveau}{Für einen Test wählt zuerst ein Signifikanzniveau $\alpha \in(0,1)$ und verlangt
$$
\sup _{\vartheta \in \Theta_0} \mathbb{P}_{\vartheta}[T \in K] \leq \alpha .
$$
	Man kontrolliert also die Wahrscheinlichkeit für einen Fehler 1. Art durch $\alpha$. Für $\alpha \downarrow$ gilt
	\begin{itemize}[leftmargin=*]
	\item Prob für Fehler 1. Art wird kleiner
	\item Verwerfungsbereich muss kleiner gewählt werden
	\item Prob für Fehler 2. Art wird grösser
	\item Macht des Tests wird kleiner
	\end{itemize}
	}
	



\DEF{8.7. (Likelihood-Quotient)}{
   Für $\vartheta_0 \in \Theta_0, \vartheta_A \in \Theta_A$ und $x_1, \ldots, x_n \in \mathbb{R}$ definieren wir den Likelihood-Quotienten durch
 $$
 R\left(x_1, \ldots, x_n ; \vartheta_0, \vartheta_A\right)=\frac{L\left(x_1, \ldots, x_n ; \vartheta_A\right)}{L\left(x_1, \ldots, x_n ; \vartheta_0\right)} \text {. }
 $$
 
 Als Konvention setzen wir $R\left(x_1, \ldots, x_n ; \vartheta_0, \vartheta_A\right)=+\infty$, wenn $L\left(x_1, \ldots, x_n ; \vartheta_0\right)=0$}

\DEF{8.8. (Likelihood-Quotienten-Test)}{
   Sei $c \geq 0$. Der Likelihood-Quotienten-Test mit Parameter $c$ ist ein Test $(T, K)$ mit Teststatistik $T=R\left(X_1, \ldots, X_n ; \vartheta_0, \vartheta_A\right)$ und Verwerfungsbereich $K=(c, \infty]$.}
   
   
   \LEM{8.9. (Neyman-Pearson-Lemma)}{
   Sei $\Theta_0=\left\{\vartheta_0\right\}$ und $\Theta_A=\left\{\vartheta_A\right\}$. Sei $(T, K)$ ein Likelihood-Quotienten-Test mit Parameter $c$ und Signifikanzniveau $\alpha^*:=\mathbb{P}_{\vartheta_0}[T \in K]$. Ist $\left(T^{\prime}, K^{\prime}\right)$ ein anderer Test mit Signifikanzniveau $\mathbb{P}_{\vartheta_0}\left[T^{\prime} \in K^{\prime}\right]=: \alpha \leq \alpha^*$, so gilt auch
 $$
 \mathbb{P}_{\vartheta_A}\left[T^{\prime} \in K^{\prime}\right] \leq \mathbb{P}_{\vartheta_A}[T \in K] .
 $$
 Das bedeutet, jeder andere Test mit kleinerem Signifikanzniveau hat auch kleinere Macht bzw. grössere Wahrscheinlichkeit für einen Fehler 2. Art.}
 
 \DEF{8.10. (Verallgemeinerung des Likelihood-Quotient)}{
Der verallgemeinerte Likelihood-Quotient ist gegeben durch
$$
 R\left(x_1, \ldots, x_n\right)=\frac{\sup _{\vartheta \in \Theta_A} L\left(x_1, \ldots, x_n ; \vartheta\right)}{\sup _{\vartheta \in \Theta_0} L\left(x_1, \ldots, x_n ; \vartheta\right)}
 $$
 oder auch
 $$
 \tilde{R}\left(x_1, \ldots, x_n\right)=\frac{\sup _{\vartheta \in \Theta_A \cup \Theta_0} L\left(x_1, \ldots, x_n ; \vartheta\right)}{\sup _{\vartheta \in \Theta_0} L\left(x_1, \ldots, x_n ; \vartheta\right)} .
 $$}



\KRZ{Z-Test}{
Gegeben:
\begin{itemize}[leftmargin=*]
\item Daten: $x_i$ und $\bar{x}=\frac{\sum^{i=1}{n} x_i }{n}$
\item Nullhypothese : $\mu_{0}$
\item Alternativhypothese :

\begin{itemize}[leftmargin=*]
\item $\mu >\mu_{0}$ (rechtsseitig)
\item $\mu <\mu_{0}$ (linksseitig)
\item $\mu \neq \mu_{0}$ (beidseitig)
\end{itemize}  
\item Signifikanzniveau : $\alpha$
\end{itemize}
\begin{enumerate}[leftmargin=*]
\item Berechne $Z$-Score
$$
Z=\frac{\bar{x}-\mu_0}{\frac{\sigma}{\sqrt{n}}}
$$
\item Berechne $z$
\begin{itemize}[leftmargin=*]
\item $z = z_{1-\alpha} = \Phi^{-1}(1-\alpha)	$ (rechtsseitig)
\item $z = z_{\alpha} = \Phi^{-1}(\alpha)	$ (linksseitig)
\item $z_{+,-} = \pm z_{\frac{\alpha}{2}} = \pm \Phi^{-1}(\frac{\alpha}{2})	$  (beidseitig)
\end{itemize}
\item Entscheide:
\begin{itemize}[leftmargin=*]
\item $Z>z \Longrightarrow \text{ verwerfe } \mu_{0}$ (rechtsseitig)
\item $Z < z \Longrightarrow \text{ verwerfe } \mu_{0}$ (linksseitig)
\item   $Z>z_{+} \vee  Z<z_{-} \Longrightarrow \text{ verwerfe } \mu_{0}$    (beidseitig)
\end{itemize}
\end{enumerate}
}




